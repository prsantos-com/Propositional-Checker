\documentclass{article}
%
%
\makeatletter
\@ifundefined{lhs2tex.lhs2tex.sty.read}%
  {\@namedef{lhs2tex.lhs2tex.sty.read}{}%
   \newcommand\SkipToFmtEnd{}%
   \newcommand\EndFmtInput{}%
   \long\def\SkipToFmtEnd#1\EndFmtInput{}%
  }\SkipToFmtEnd

\newcommand\ReadOnlyOnce[1]{\@ifundefined{#1}{\@namedef{#1}{}}\SkipToFmtEnd}
\usepackage{amstext}
\usepackage{amssymb}
\usepackage{stmaryrd}
\DeclareFontFamily{OT1}{cmtex}{}
\DeclareFontShape{OT1}{cmtex}{m}{n}
  {<5><6><7><8>cmtex8
   <9>cmtex9
   <10><10.95><12><14.4><17.28><20.74><24.88>cmtex10}{}
\DeclareFontShape{OT1}{cmtex}{m}{it}
  {<-> ssub * cmtt/m/it}{}
\newcommand{\texfamily}{\fontfamily{cmtex}\selectfont}
\DeclareFontShape{OT1}{cmtt}{bx}{n}
  {<5><6><7><8>cmtt8
   <9>cmbtt9
   <10><10.95><12><14.4><17.28><20.74><24.88>cmbtt10}{}
\DeclareFontShape{OT1}{cmtex}{bx}{n}
  {<-> ssub * cmtt/bx/n}{}
\newcommand{\tex}[1]{\text{\texfamily#1}}	% NEU

\newcommand{\Sp}{\hskip.33334em\relax}
\newcommand{\NB}{\textbf{NB}}
\newcommand{\Todo}[1]{$\langle$\textbf{To do:}~#1$\rangle$}

\EndFmtInput
\makeatother
%
%
%
%
%
%
% This package provides two environments suitable to take the place
% of hscode, called "plainhscode" and "arrayhscode". 
%
% The plain environment surrounds each code block by vertical space,
% and it uses \abovedisplayskip and \belowdisplayskip to get spacing
% similar to formulas. Note that if these dimensions are changed,
% the spacing around displayed math formulas changes as well.
% All code is indented using \leftskip.
%
% Changed 19.08.2004 to reflect changes in colorcode. Should work with
% CodeGroup.sty.
%
\ReadOnlyOnce{polycode.fmt}%
\makeatletter

\newcommand{\hsnewpar}[1]%
  {{\parskip=0pt\parindent=0pt\par\vskip #1\noindent}}

% can be used, for instance, to redefine the code size, by setting the
% command to \small or something alike
\newcommand{\hscodestyle}{}

% The command \sethscode can be used to switch the code formatting
% behaviour by mapping the hscode environment in the subst directive
% to a new LaTeX environment.

\newcommand{\sethscode}[1]%
  {\expandafter\let\expandafter\hscode\csname #1\endcsname
   \expandafter\let\expandafter\endhscode\csname end#1\endcsname}

% "compatibility" mode restores the non-polycode.fmt layout.

\newenvironment{compathscode}%
  {\par\noindent
   \advance\leftskip\mathindent
   \hscodestyle
   \let\\=\@normalcr
   \let\hspre\(\let\hspost\)%
   \pboxed}%
  {\endpboxed\)%
   \par\noindent
   \ignorespacesafterend}

\newcommand{\compaths}{\sethscode{compathscode}}

% "plain" mode is the proposed default.
% It should now work with \centering.
% This required some changes. The old version
% is still available for reference as oldplainhscode.

\newenvironment{plainhscode}%
  {\hsnewpar\abovedisplayskip
   \advance\leftskip\mathindent
   \hscodestyle
   \let\hspre\(\let\hspost\)%
   \pboxed}%
  {\endpboxed%
   \hsnewpar\belowdisplayskip
   \ignorespacesafterend}

\newenvironment{oldplainhscode}%
  {\hsnewpar\abovedisplayskip
   \advance\leftskip\mathindent
   \hscodestyle
   \let\\=\@normalcr
   \(\pboxed}%
  {\endpboxed\)%
   \hsnewpar\belowdisplayskip
   \ignorespacesafterend}

% Here, we make plainhscode the default environment.

\newcommand{\plainhs}{\sethscode{plainhscode}}
\newcommand{\oldplainhs}{\sethscode{oldplainhscode}}
\plainhs

% The arrayhscode is like plain, but makes use of polytable's
% parray environment which disallows page breaks in code blocks.

\newenvironment{arrayhscode}%
  {\hsnewpar\abovedisplayskip
   \advance\leftskip\mathindent
   \hscodestyle
   \let\\=\@normalcr
   \(\parray}%
  {\endparray\)%
   \hsnewpar\belowdisplayskip
   \ignorespacesafterend}

\newcommand{\arrayhs}{\sethscode{arrayhscode}}

% The mathhscode environment also makes use of polytable's parray 
% environment. It is supposed to be used only inside math mode 
% (I used it to typeset the type rules in my thesis).

\newenvironment{mathhscode}%
  {\parray}{\endparray}

\newcommand{\mathhs}{\sethscode{mathhscode}}

% texths is similar to mathhs, but works in text mode.

\newenvironment{texthscode}%
  {\(\parray}{\endparray\)}

\newcommand{\texths}{\sethscode{texthscode}}

% The framed environment places code in a framed box.

\def\codeframewidth{\arrayrulewidth}
\RequirePackage{calc}

\newenvironment{framedhscode}%
  {\parskip=\abovedisplayskip\par\noindent
   \hscodestyle
   \arrayrulewidth=\codeframewidth
   \tabular{@{}|p{\linewidth-2\arraycolsep-2\arrayrulewidth-2pt}|@{}}%
   \hline\framedhslinecorrect\\{-1.5ex}%
   \let\endoflinesave=\\
   \let\\=\@normalcr
   \(\pboxed}%
  {\endpboxed\)%
   \framedhslinecorrect\endoflinesave{.5ex}\hline
   \endtabular
   \parskip=\belowdisplayskip\par\noindent
   \ignorespacesafterend}

\newcommand{\framedhslinecorrect}[2]%
  {#1[#2]}

\newcommand{\framedhs}{\sethscode{framedhscode}}

% The inlinehscode environment is an experimental environment
% that can be used to typeset displayed code inline.

\newenvironment{inlinehscode}%
  {\(\def\column##1##2{}%
   \let\>\undefined\let\<\undefined\let\\\undefined
   \newcommand\>[1][]{}\newcommand\<[1][]{}\newcommand\\[1][]{}%
   \def\fromto##1##2##3{##3}%
   \def\nextline{}}{\) }%

\newcommand{\inlinehs}{\sethscode{inlinehscode}}

% The joincode environment is a separate environment that
% can be used to surround and thereby connect multiple code
% blocks.

\newenvironment{joincode}%
  {\let\orighscode=\hscode
   \let\origendhscode=\endhscode
   \def\endhscode{\def\hscode{\endgroup\def\@currenvir{hscode}\\}\begingroup}
   %\let\SaveRestoreHook=\empty
   %\let\ColumnHook=\empty
   %\let\resethooks=\empty
   \orighscode\def\hscode{\endgroup\def\@currenvir{hscode}}}%
  {\origendhscode
   \global\let\hscode=\orighscode
   \global\let\endhscode=\origendhscode}%

\makeatother
\EndFmtInput
%
\begin{document}

This is my propostional parser using Parsec (parsec2 package)

\begin{tabbing}\tt
~module~PropParser~where\\
\tt ~\\
\tt ~import~Text\char46{}ParserCombinators\char46{}Parsec~\char45{}\char45{}~\char58{}set~\char45{}ignore\char45{}package~parsec\char45{}3\char46{}1\char46{}0\\
\tt ~import~Text\char46{}ParserCombinators\char46{}Parsec\char46{}Expr\\
\tt ~import~qualified~Text\char46{}ParserCombinators\char46{}Parsec\char46{}Token~as~T\\
\tt ~import~Text\char46{}ParserCombinators\char46{}Parsec\char46{}Language~\char40{}haskellDef\char41{}\\
\tt ~import~Text\char46{}ParserCombinators\char46{}Parsec\char46{}Char\\
\tt ~\\
\tt ~import~Myitautology\\
\tt ~import~Char
\end{tabbing}

\begin{tabbing}\tt
~\char45{}\char45{}~The~Parser\\
\tt ~mainParser~\char61{}~do~whiteSpace\\
\tt ~~~~~~~~~~~~~~~~~e~\char60{}\char45{}~expr\\
\tt ~~~~~~~~~~~~~~~~~eof\\
\tt ~~~~~~~~~~~~~~~~~return~e\\
\tt ~\\
\tt ~expr~~~~\char61{}~buildExpressionParser~table~term\\
\tt ~~~~~~~~~\char60{}\char63{}\char62{}~\char34{}expression\char34{}\\
\tt ~\\
\tt ~term0~~~~\char61{}~~parens~expr\\
\tt ~~~~~~~~~\char60{}\char124{}\char62{}~var\\
\tt ~~~~~~~~~\char60{}\char63{}\char62{}~\char34{}simple~proposition\char34{}\\
\tt ~\\
\tt ~term~~~~~\char61{}~do\\
\tt ~~~~~~~~~t~\char60{}\char45{}~term0\\
\tt ~~~~~~~~~whiteSpace\\
\tt ~~~~~~~~~return~t\\
\tt ~\\
\tt ~table~~~\char58{}\char58{}~OperatorTable~Char~\char40{}\char41{}~Prop\\
\tt ~table~~~\char61{}~\char91{}~\char91{}prefix~\char34{}\char126{}\char34{}~Not~\char93{}\\
\tt ~~~~~~~~~~~\char44{}~\char91{}binary~\char34{}\char38{}\char34{}~And~AssocLeft\char44{}~binary~\char34{}v\char34{}~Or~AssocLeft~\char93{}\\
\tt ~~~~~~~~~~~\char44{}~\char91{}binary~\char34{}\char61{}\char62{}\char34{}~Imply~AssocLeft\char44{}~binary~\char34{}\char60{}\char61{}\char62{}\char34{}~Equiv~AssocNone~\char93{}\\
\tt ~~~~~~~~~~~\char93{}\\
\tt ~\\
\tt ~binary~~name~fun~assoc~\char61{}~Infix~\char40{}do\char123{}~reservedOp~name\char59{}~whiteSpace\char59{}~return~fun~\char125{}\char41{}~assoc\\
\tt ~prefix~~name~fun~~~~~~~\char61{}~Prefix~\char40{}do\char123{}~reservedOp~name\char59{}~whiteSpace\char59{}~return~fun~\char125{}\char41{}\\
\tt ~postfix~name~fun~~~~~~~\char61{}~Postfix~\char40{}do\char123{}~reservedOp~name\char59{}~whiteSpace\char59{}~return~fun~\char125{}\char41{}\\
\tt ~\\
\tt ~isVar~~~~~~~~~~~~~~~~~~~\char58{}\char58{}~Char~\char45{}\char62{}~Bool\\
\tt ~isVar~c~~~~~~~~~~~~~~~~~\char61{}~~isAlpha~c~\char38{}\char38{}~c~\char47{}\char61{}~\char39{}v\char39{}\\
\tt ~\\
\tt ~var~~~~~~~~~~~~~~~~~~~~~\char58{}\char58{}~Parser~Prop\\
\tt ~var~~~~~~~~~~~~~~~~~~~~~\char61{}~~fmap~Var~\char36{}~satisfy~isVar\\
\tt ~\\
\tt ~evalProp~~~~~~~~~~~~~~~\char58{}\char58{}~String~\char45{}\char62{}~String~\char45{}\char62{}~Rests~\char45{}\char62{}~String\\
\tt ~evalProp~x1~x2~rs~~~~~~~~~~~\\
\tt ~~~~~~~~~~~~~~~~~\char61{}~~case~\char40{}parse~mainParser~\char34{}\char34{}~x1\char41{}~of\\
\tt ~~~~~~~~~~~~~~~~~~~~~~~~~Left~err1~\char45{}\char62{}~show~err1\\
\tt ~~~~~~~~~~~~~~~~~~~~~~~~~Right~~p1~\char45{}\char62{}~case~\char40{}parse~mainParser~\char34{}\char34{}~x2\char41{}~of\\
\tt ~~~~~~~~~~~~~~~~~~~~~~~~~~~~~~~~~~~~~~~~~Left~err2~\char45{}\char62{}~show~err2\\
\tt ~~~~~~~~~~~~~~~~~~~~~~~~~~~~~~~~~~~~~~~~~Right~~p2~\char45{}\char62{}~show~\char40{}propMachine~p1~p2~rs\char41{}\\
\tt ~\\
\tt ~\char45{}\char45{}~This~will~convert~the~first~string~to~a~Prop~and~allow~disagree~to~work\\
\tt ~evalDisagree~~~~~~~~~~~~\char58{}\char58{}~String~\char45{}\char62{}~String~\char45{}\char62{}~Rests~\char45{}\char62{}~String\\
\tt ~evalDisagree~x1~x2~rs~~~~\char61{}~~case~\char40{}parse~mainParser~\char34{}\char34{}~x1\char41{}~of\\
\tt ~~~~~~~~~~~~~~~~~~~~~~~~~~~~~~~~~Left~err1~\char45{}\char62{}~show~err1\\
\tt ~~~~~~~~~~~~~~~~~~~~~~~~~~~~~~~~~Right~p1~\char45{}\char62{}~case~\char40{}parse~mainParser~\char34{}\char34{}~x2\char41{}~of\\
\tt ~~~~~~~~~~~~~~~~~~~~~~~~~~~~~~~~~~~~~~~~~~~~~~~~~Left~err2~\char45{}\char62{}~show~err2\\
\tt ~~~~~~~~~~~~~~~~~~~~~~~~~~~~~~~~~~~~~~~~~~~~~~~~~Right~~p2~\char45{}\char62{}~disagree~p1~p2~rs\\
\tt ~\\
\tt ~\char45{}\char45{}~Restriction~parser\\
\tt ~getRests~~~~~~~~~~~~~~~~\char58{}\char58{}~\char91{}String\char93{}~\char45{}\char62{}~Rests~\char45{}\char62{}~Either~String~Rests\\
\tt ~getRests~\char91{}\char93{}~~~~~ps~~~~~~\char61{}~~Right~ps\\
\tt ~getRests~\char40{}r\char58{}rs\char41{}~ps~~~~~~\char61{}~~case~\char40{}parse~mainParser~\char34{}\char34{}~r\char41{}~of\\
\tt ~~~~~~~~~~~~~~~~~~~~~~~~~~~~~~~~~Left~err~\char45{}\char62{}~Left~\char34{}hello\char34{}\\
\tt ~~~~~~~~~~~~~~~~~~~~~~~~~~~~~~~~~Right~~p~\char45{}\char62{}~Right~\char91{}p\char93{}\\
\tt ~\\
\tt ~removeEmpty~xs~\char61{}~filter~\char40{}\char47{}\char61{}~\char34{}\char34{}\char41{}~xs\\
\tt ~\\
\tt ~\char45{}\char45{}~The~lexer\\
\tt ~lexer~~~~~~~\char61{}~T\char46{}makeTokenParser~haskellDef\\
\tt ~lexeme~~~~~~\char61{}~T\char46{}lexeme\\
\tt ~\\
\tt ~parens~~~~~~\char61{}~T\char46{}parens~lexer\\
\tt ~natural~~~~~\char61{}~T\char46{}natural~lexer\\
\tt ~reservedOp~~\char61{}~T\char46{}reservedOp~lexer\\
\tt ~whiteSpace~~\char61{}~T\char46{}whiteSpace~lexer
\end{tabbing}
\end{document}
